\renewcommand{\documentname}{Report}

\chapter{Report}

\section{Summary of the document}

\textbf{TODO: EXPLAIN MORE DETAILS ABOUT THE SECTIONS}

This document presents all the important information regarding the \textit{\tfg} end-of-degree thesis.

First, a report of the project is given. It functions as a general summary of the work done. This is followed by the introduction to the project, which gives a justification for the project, lays down the objectives and analyses the past and present situation.

After that, the theoretical sections of the project are presented. They consist of a number of definitions of key concepts, a formal definition of the problem to solve and a detailed explanation of the proposed solution.

Then the project planning and budget overview are shown.

Subsequent to the project planning rundown follow the software engineering chapters, which include the system analysis, design and implementation and the test development. 

After outlining the specifications of the system, the results of the experiments conducted are listed.

The system manuals for the programmer and the user are handed.

Finally, the conclusions and future work are listed, along with the final budgets, the bibliography, annexes and the source code.

It is important to note that the structure of the contents for this document is done following the criteria and recommendations of the template document for Degree's and Master's Thesis of the School of Computing Engineering of Oviedo (version 1.4) \cite{doc-template} by J. M. Redondo. However, some additional chapters were introduced by the author in order to capture the particularity of the work carried out, inspired by the research of G. de la Cruz \cite{metaheuristics-groups}.


\section{Purpose}\label{purpose}

This project aims to help the personnel of the School of Computing Engineering at the University of Oviedo manage their classrooms. It will address two main functionalities, the automation of the process of assigning classrooms to all the groups of a given semester (starting from scratch or using a previous partial or total assignment), and a tool that searches for gaps in a previous set of assignments for single or multi-day events in one or more classes.

The implementation of this system is intended to assist in the work of the supervisor for this process, and provide an efficient and flexible tool that expands the possibilities of such work. To do so, the program executes two algorithms, a genetic algorithm guided by a greedy algorithm. For a more detailed view on these algorithms the reader might refer to \ref{theory}. Once the assignments have been calculated, the system will allow the users to find classrooms to hold specific events in the middle of the semester.

Along with the system, the system manuals are submitted. These have the purpose of teaching how to install, use, maintain and extend the system. Apart from the manuals, another tool to generate the necessary files for the program is handed.


\section{Scope}

The project needs to formally define the problem of assigning classrooms of the School to all the groups of the semester, conduct a study on the problem and propose a solution. 

A development of a software prototype that solves the problem is planned, designed, implemented and tested. This prototype will solve the two main functionalities indicated in \ref{purpose} and will consist of a command line application that takes input data in plain text files and outputs the solution to plain text files. The program is configured by different configuration files depending on the functionality being executed. An experimental study on the results of the software system is carried out, finding the most fitting default values for the configuration files. The project also contains the system manuals of the application, which consist of the installation, usage, user and programmer manuals. 

Finally, an additional tool for automating the creation of the input files of the software is given. It uses a format agreed with the client and will use the same technical specifications of the main prototype, like the programming language and the development environment.


