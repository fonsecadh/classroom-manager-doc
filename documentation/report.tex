\renewcommand{\documentname}{Report}

\chapter{Report}

\section{Motivation}

\epigraph{First, solve the problem. Then, write the code.}{\textit{John Johnson}}

One of the things I learned in my university years is that when you are presented with a coding problem, be it an exam or a project, is good to step back and try to solve it on paper. This project embodies that sentiment.

This is a problem you need to solve \textit{before} writing the code, first of all, because the code is a \textit{prototype} of the proposed solution, and all its elements \textit{depend} on the algorithm in the first place. Input data? Data structures? Good luck with those if you start typing away without thinking. 

However, even if you do solve the problem, there is no guarantee that it will work in practice. Maybe the computations are so complex that the time the program takes to solve the problem makes it unusable, or you simply cannot give useful solutions without creating a lot of conflicting assignments. This uncertainty factor is undoubtedly a characteristic of the project that sets it apart from others I have previously undertaken.

Lastly, it is important to mention that the success of this project will make someone's job easier, so prototype or not, the system needs to work and must have as the primary goal the experience of the user.


\section{Purpose}\label{purpose}

This project aims to help the personnel of the School of Computing Engineering at the University of Oviedo manage their classrooms. It will address two main functionalities, the automation of the process of assigning classrooms to all the groups of a given semester (starting from scratch or using a previous partial or total assignment), and a tool that searches for gaps in a previous set of assignments for single or multi-day events in one or more classes.

The implementation of this system is intended to assist in the work of the supervisor for this process, and provide an efficient and flexible tool that expands the possibilities of such work. To do so, the program executes two algorithms, a genetic algorithm guided by a greedy algorithm. For a more detailed view on these algorithms the reader might refer to \ref{theory}. Once the assignments have been calculated, the system will allow the users to find classrooms to hold specific events in the middle of the semester.

Along with the system, the system manuals are submitted. These have the purpose of teaching how to install, use, maintain and extend the system. Apart from the manuals, another tool to generate the necessary files for the program is handed.


\section{Scope}

The project needs to formally define the problem of assigning classrooms of the School to all the groups of the semester, conduct a study on the problem and propose a solution. 

A development of a software prototype that solves the problem is planned, designed, implemented and tested. This prototype will solve the two main functionalities indicated in \ref{purpose} and will consist of a command line application that takes input data in plain text files and outputs the solution to plain text files. The program is configured by different configuration files depending on the functionality being executed. An experimental study on the results of the software system is carried out, finding the most fitting default values for the configuration files. The project also contains the system manuals of the application, which consist of the installation, usage, user and programmer manuals. 

Finally, an additional tool for automating the creation of the input files of the software is given. It uses a format agreed with the client and will use the same technical specifications of the main prototype, like the programming language and the development environment.


