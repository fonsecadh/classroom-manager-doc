\renewcommand{\documentname}{Report}

\chapter{Report}

\section{Introduction}

This document presents all the important information regarding the \tfg final degree project.

First, a report of the project is given. It functions as a general summary of the work done. The current and past situation in relation with the problem to solve are outlined, followed by briefly describing the proposed solution and the summary of the planning and budget.

Then, all the annexes are presented. They are geared towards the technical reader, and contain the initial documentation for the project, the complete system analysis and design, how the planning was elaborated, the studies related to the algorithms to be used for the implementation and finally the manuals for users and programmers of the final program.

The annexes are followed by the complete list of the requirements, functional and non-functional, and the internal and client budget breakdown.

Finally, the conclusions, future work and bibliography complete the document.

It is important to note that the organization of the contents for this document is done following the criteria and recommendations of the UNE 157801 norm for the development of information systems projects.

\section{Purpose}

This project aims to create a program that automates the process of assigning classrooms to all the groups of a given semester.

The implementation of this system is intended to assist in the work of the supervisor for this process, and provide an efficient and flexible tool that expands the possibilities of such work.

To do so, the program executes two algorithms, a genetic algorithm whose fitness function uses a greedy algorithm. For a more detailed view on these algorithms the reader might refer to \ref{alg-studies}.

Along with the system, the user and programmer manuals are submitted. These have the purpose of teaching how to use, maintain and extend the system.

\section{Background}

At the beginning of each semester, the School of Computing Engineering of the University of Oviedo opens a process in which the person in charge takes the list of groups for the semester, their schedules and the list of classrooms, and performs a manual compilation of all the assignments.

There are some other similar procedures, like the creation of the exam timetable or the assignments of enrolled students to subject groups. However, some are not manual, but automated by a system, like the previously mentioned procedure of assigning students to groups.

Seeing the potential of such tool, the author of this project was given the task of automating the assignment of classrooms to subject groups. 

\section{Description of the current situation}

As explained in the previous section, the school has a supervisor for the process of assigning classrooms to groups.

This procedure is done after configuring the student groups for the semester and knowing their schedules.

Even though it is a manual process, the supervisor does not start making the assignments from scratch. First, they have the knowledge of previous years, and then they have a list of preferences or premade assignments. For example, certain laboratories can only be assigned to specific groups, like the ones from the Electronic Technology of Computers subject.

The system described in this document preserves these sources of information and builds on top of them.

\section{Standards and references}
\section{Definitions and abbreviations}
\section{Initial requirements}
\section{Assumptions and restrictions}
\section{Study of alternatives and feasibility}
\section{Description of the proposed solution}
\section{Risk analysis}
\section{Time planning}
\section{Budget summary}
\section{Prioritisation of project documents}

