\renewcommand{\documentname}{Conclusions and future work}

\chapter{Conclusions and future work}

\section{Final conclusions}

All the objectives set for the project have been met. A study and formalisation of the problem has been carried out and the problem was solved, both theoretically and experimentally, thanks to a software tool.

The files necessary for the software to work have been designed and created. They are provided in the annexes, many of which can be reused by the user of the application. In addition, an experimental study has been conducted on the developed solution in order to provide the customer with the default configuration of the system to ensure quality results in various possible scenarios.

The developed system is licensed under a free software licence and may be modified and distributed by users who wish to do so.

With all this in mind, the project comes to a close with satisfaction on my part, although the work does not end here. In the next section, some possible lines for the continuation of the work developed in this project are explained.


\section{Future work}

Although the objectives of the project have been met, nothing changes the fact that the software designed in this document is a prototype. Of course, a fully usable prototype with interesting functionalities, but there is room for improvement. Listed below are some lines of research and development to be pursued on the basis of this project.


\begin{itemize}
    \item \textbf{Expansion of the class search functionality to return results that support multiple events}. Currently the user can browse the results looking for free classes at the same time, but perhaps it would be nice to have a feature that allows you to choose the number of classes you are seeking for an event that are free at the same time.
    \item \textbf{Increased validation of input files}. The current validation is perfectly valid, and can help the user to find most of the problems. However, more specific checks such as the formatting of a group/class/subject code, that the end dates of queries are not less than the start dates, etc, would provide a better user experience.
    \item \textbf{Possibility of assigning different classes to each group depending on the day}. This is \textit{very complicated}, as it affects the main pillar of the whole theory that a group can only have one classroom associated with it. However, if one wanted to rethink the problem from scratch, this approach could be taken into account.
    \item \textbf{Rewrite the code in a language such as C or Go, more focused on algorithms}. It should be noted that the code described here, as mentioned above, is only a prototype. It can and does work correctly, but if new functionalities appear that greatly affect the original design, one should not be afraid to rewrite the code, and in that case, another language could be used.
\end{itemize}


