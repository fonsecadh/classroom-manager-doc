\renewcommand{\documentname}{Overview}

\chapter{Overview}

This document presents all the important information regarding the \textit{\tfg} end-of-degree thesis.

It is important to note that the structure of the contents for this document is done following the criteria and recommendations of the template document for Degree's and Master's Thesis of the School of Computing Engineering of Oviedo (version 1.4) by José Manuel Redondo \cite{redondotemplate}. However, some additional chapters were introduced in order to capture the particularity of the work carried out, inspired by the work of Gonzalo de la Cruz \cite{delacruz18metaheuristics}.

This document is structured in the following chapters:

\begin{description}

    \item \textbf{Introduction}. Here we explain in a simple way the problem we want to solve, what reasons are behind the development of the project and give a description of the current situation of the School with regard to this and other similar problems. We also provide a broad outline of the scope for the project and what objectives need to be met for the project to succeed.

    \item \textbf{Theoretical aspects}. This chapter delves into the theory supporting the developed system. An example of an assignment problem is presented and solved using a greedy algorithm. Then, an in-depth study of heuristics and metaheuristics is carried out, covering a wide range of evolutionary computation algorithms. Finally, an explanation is given on how to combine the greedy and genetic algorithms to solve the example problem, which mirrors on a small scale the overall work done for the project.

    \item \textbf{Problem definition}. Here the formulation of the problem as an assignment problem is elaborated. We present the information we have of the problem and simultaneously carry out a structural analysis of its components. 

    \item \textbf{Proposed solution}. This chapter lays down in detail the proposed solution to the previously defined assignment problem. It is in this chapter where we define the search space of the problem, basic concepts to understand the behaviour of the algorithms and the techniques previously identified in the theoretical aspects section that we use to solve the real problem. 

    \item \textbf{Project planning and budget overview}. This outlines the project planning and the internal and client budgets. The Work Breakdown Structure (WBS) on a larger scale can be found in the annexes and the fully detailed budget is given in its own section at the end of the document. 

    \item \textbf{Analysis}. An analysis of the system, with the system requirements, draft diagrams, use cases and the test plan specification. 

    \item \textbf{System design}. The technical details of the system analysed in the previous chapter. Here we include the finalised diagrams, as well as the arquitecture of the system and the in-depth test plan.

    \item \textbf{System implementation}. Details of the development of the software. The programming languages, standards and tools used to code the system, and all the relevant information gathered in the process of creating the system.

    \item \textbf{Test development}. A rundown of all the testing done for the system, with explanations for every test and the obtained results. 

    \item \textbf{Experimental results}. The conclusions reached after experimenting with different input data and the proposed configuration for the parameters of the genetic algorithm. 

    \item \textbf{System manuals}. All system manuals are provided, with explanations and guided steps in the required level of detail to help the target readers of each manual to complete their work..

    \item \textbf{Conclusions and future work}. I express here the findings that I have obtained through the development of this project and also point out a series of project-related topics that may be carried out in the future. 

    \item \textbf{Budget}. The full details for the elaboration of the internal and client budgets are shown, with all the intermediate steps that we took to calculate them. 

    \item \textbf{Annexes}. The glossary of definitions and abbreviations and a small commentary on the submitted files. In addition to this, you can find here additional information on the project that was not necessary to elaborate on in the different chapters. 

\end{description}

