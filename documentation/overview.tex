\renewcommand{\documentname}{Overview}

\chapter{Overview}

This document presents all the important information regarding the \textit{\tfg} end-of-degree thesis.

It is important to note that the structure of the contents for this document is done following the criteria and recommendations of the template document for Degree's and Master's Thesis of the School of Computing Engineering of Oviedo (version 1.4) by Redondo \cite{redondotemplate}. However, some additional chapters were introduced in order to capture the particularity of the work carried out, inspired by the research of de la Cruz \cite{delacruz18metaheuristics}.

\begin{description}

    \item \textbf{Introduction}. Here we explain in a simple way the problem we want to solve, what reasons are behind the development of the project and give a description of the current situation of the School with regard to this and other similar problems. The scope and goals of the project are also discussed.

    \item \textbf{Theoretical aspects}. The first chapter delving into the theory supporting the developed system. One example of an assignment problem is presented and solved using a greedy algorithm and a genetic algorithm, which helps to better internalize the concepts.

    \item \textbf{Problem definition}. Here the formulation of the problem as an assignment problem is elaborated. 

    \item \textbf{Proposed solution}. This chapter lays down in detail the proposed solution to the previously defined assignment problem by means of a genetic and greedy algorithms.

    \item \textbf{Project planning and budget overview}. For the planning, the Gantt chart of the project is shown, as well as the work breakdown structure. The internal and client budgets are presented, but not how they were calculated. That goes in the budget section.

    \item \textbf{Analysis}. An analysis of the system, with the system requirements, draft diagrams, use cases and the test plan specification. 

    \item \textbf{System design}. The technical details of the system analysed in the previous section. Here we include the finalised diagrams, as well as the arquitecture of the system and the in-depth test plan.

    \item \textbf{System implementation}. Details of the development of the software. The programming languages, standards and tools used to code the system, and all the relevant information gathered in the process of creating the system.

    \item \textbf{Test development}. A rundown of all the testing done for the system, with explanations for every test and the obtained results. 

    \item \textbf{Experimental results}. The conclusions reached after experimenting with different input data and the optimal configuration for the genetic algorithm's parameters. 

    \item \textbf{System manuals}. All the manuals for the system, with screenshots and guided steps meant to help the target audience for each document. 

    \item \textbf{Conclusions and future work}. The conclusions after the implementation of the project are given, as well as a list of possible improvements and new functionalities for the prototype. 

    \item \textbf{Budget}. Here we give the full details for the elaboration of the internal and client budgets, with all the intermediate steps that led us to that result. 

    \item \textbf{Annexes}. The glossary of definitions and abbreviations and a small commentary on the submitted files.

    \item \textbf{Source code}. The \textit{javadoc} of the software developed. 

\end{description}

