\renewcommand{\documentname}{Introduction}

\chapter{Introduction}


\section{Project justification}

\section{Project goals}

\section{Situation overview}

\subsection{Background}

At the beginning of each semester, the School of Computing Engineering of the University of Oviedo opens a process in which the person in charge takes the list of groups for the semester, their schedules and the list of classrooms, and performs a manual compilation of all the assignments.

There are some other similar procedures, like the creation of the exam timetable or the assignments of enrolled students to subject groups. However, some are not manual, but automated by a system, like the previously mentioned procedure of assigning students to groups.

Seeing the potential of such tool, the author of this project was given the task of automating the assignment of classrooms to subject groups. 

\subsection{Description of the current situation}

As explained in the previous section, the school has a supervisor for the process of assigning classrooms to groups.

This procedure is done after configuring the student groups for the semester and knowing their schedules.

Even though it is a manual process, the supervisor does not start making the assignments from scratch. First, they have the knowledge of previous years, and then they have a list of preferences or premade assignments. For example, certain laboratories can only be assigned to specific groups, like the ones from the Electronic Technology of Computers subject.

The system described in this document preserves these sources of information and builds on top of them.


