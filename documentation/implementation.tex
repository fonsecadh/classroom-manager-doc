\renewcommand{\documentname}{System implementation}

\chapter{System implementation}


\section{License and references}

\subsection{License}

The software of this project is licensed under the GNU General Public License v2.0.

\subsection{References}

\begin{itemize}
    \item \textit{Java Code Conventions.} Set of guidelines and conventions for programmers to consider when using the Java programming language.

    \item \textit{Linux kernel coding style.} \footnote{Available at \url{https://www.kernel.org/doc/Documentation/process/coding-style.rst}} Set of guidelines and conventions for programmers to consider when programming in the Linux kernel. The stylistic choices in this guide have, for the most part, been the ones adopted for formatting the prototype code, as we believe that the readability of the code is preferable to the Java Code Conventions guidelines.
\end{itemize}



\section{Programming languages}

The use of C or Java for programming the system was discussed. In the end we opted for Java for two reasons. The first is simple, I, the developer, have more experience in Java than in C (although I have used both in this School). Perhaps C would be a suitable language to implement the algorithms described in this document more efficiently, but the extra time I would have to spend learning the language in an advanced way makes it unmanageable for this project. The second reason is also obvious. If this project is to be continued by our colleagues at the School, it would be preferable if it were written in the language that has been learnt throughout the degree courses, i.e, Java.

Apart from Java, the experimentation phase has made use of the Python language to extract the data from the executions and to present this data in a useful manner for further analysis.

\section{Tools and programs used in development}

\subsection{Git}

Git is a distributed version control software created by Linus Torvalds. It is used in this project to control changes to the code and documentation thanks to two repositories hosted on Github servers. 

\subsection{Vim}

Vim is a text editor created by Bram Moolenaar. In this project Vim is used to write the documentation, define the system files and any other non-Java code.

\subsection{\LaTeX}

\LaTeX is a software system used for creating documents. We made use of this technology to create the documentation from scratch, inspired by other templates such as those mentioned at the beginning of the document. The main advantage of using \LaTeX to create the documents is the possibility of integrating it into a version control system such as Git, which we have done in this project. 

\subsection{Eclipse}

Eclipse is an integrated development environment (IDE) for Java. The vast majority of the system was programmed using this tool, employing its debugging tools and a plugin that emulates the behaviour of Vim in the Eclipse editor.

\subsection{PlantUML}

PlantUML is a UML diagram generator derived from plain text. It has been used for all diagrams in the documentation, as well as for creating the WBS and Gantt charts.


