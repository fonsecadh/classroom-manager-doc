\renewcommand{\documentname}{Problem definition}

\chapter{Problem definition}

The School of Computing Engineering of Oviedo must find a classroom for each group of a given semester. A classroom can be a laboratory or a theory classroom. Groups belong to subjects. They can be taught in English or Spanish, and have three types. Laboratory, theory and seminar groups. Laboratory groups can only go to laboratories and theory and seminar groups can only go to theory classrooms. A group can attend classes weekly, every two weeks or on a non-trivial pattern, and may be taught on one or several days. A group cannot be assigned to a classroom of a different type (for example, a laboratory group must not go to a theory classroom) or to a classroom whose capacity is less than the number of students in the group.

Laboratory groups of the same subject must all attend the same laboratory classroom, and if not possible, at least minimise the number of laboratories assigned to them. Groups of the same name and course work in the same way, but being assigned to theory classrooms \footnote{All the groups in the School have the format \textit{SUBJECT.TYPE.NAME}. For example the group \textit{Com.T.1} refers to \textit{theory} group \textit{1} of the \textit{Computability} subject. So all theory groups named 1 would be assigned to the same theory classroom, if possible.}. English and Spanish groups should go to different classrooms.

