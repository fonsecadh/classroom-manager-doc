\renewcommand{\documentname}{System manuals}

\chapter{System manuals}


\section{Installation manual}

\subsection*{SETUP}

In order to be able to execute the \textbf{classmanager} utility, two things are required. The first is Java 8 or higher, the second is the creation of the runtime environment. This environment simply consists of having the utility at the same level (in terms of folders) as a folder that the user must create, called \textbf{classmanager\_log}.

\subsection*{FOLDER TREE}

When we run the jar file, we will do it from this folder. If the configuration files have relative paths, they must be relative to the jar file. An example of this is shown below.

\begin{description}
    \item \textbf{Classmanager/} (FOLDER)
        \begin{description}
            \item \textit{classmanager.jar} (Executable JAR)
            \item \textbf{classmanager\_log/} (Required FOLDER)
                \begin{description}
                    \item \textit{Some logs...}
                \end{description}
            \item \textbf{inputfiles/} (User FOLDER)
                \begin{description}
                    \item \textit{classrooms.csv}
                    \item \textit{...}
                \end{description}
            \item \textbf{output/} (User FOLDER)
            \item \textbf{config/} (User FOLDER)
                \begin{description}
                    \item \textit{myConfig.properties}
                \end{description}
        \end{description}
\end{description}

We have the previously described folder in our computer. Let's say we are creating the configuration file \textit{myConfig.properties}, and we want to give a path to the CLASSROOMS\_FILE\_PATH property. Should we specify the relative path from the configuration file or from the jar? As said before, the path is \textbf{relative to the jar file}, so in our case, \verb|CLASSROOMS_FILE_PATH = ./inputfiles/classrooms.csv|.



\section{Execution manual}


\subsection*{NAME}

\begin{description}
    \item classmanager - manages the classrooms and groups of the School of Computing Engineering of Oviedo.
\end{description}


\subsection*{SYNOPSIS}

\begin{description}
    \item \textbf{classmanager} \textit{OPTION} [\textit{FILE...}]
\end{description}


\subsection*{DESCRIPTION}

The \textit{classmanager} utility assigns classes to groups in the School for a given semester. In addition, it allows queries to be made on the classrooms to obtain those available in a range of dates and schedules. Finally, the utility has a built-in tool that allows for automatic conversion of the files currently used by the school into files compatible with the system. 

A \textit{FILE} represents a configuration file. Each option requires a set of properties with values of different types. As long as these properties are given to the utility, the user is free to split them in as many configuration files as they want. If a file is not provided, or if the file cannot be parsed due to lack of permissions, the utility will notify the user of the error. The same goes for the required data not submitted in the configuration files. 


\subsection*{OPTIONS}

\subsubsection*{Generic Program Information}

\begin{description}
    \item \textbf{-h, -{}-help} 
        \begin{description}
            \item Output a usage message and exit.
        \end{description}
    \item \textbf{-v, -{}-version} 
        \begin{description}
            \item Output the version of \textbf{classmanager} and exit.
        \end{description}
\end{description}

\subsubsection*{Functionalities}

\begin{description}
    \item \textbf{-a, -{}-algorithm} 
        \begin{description}
            \item Perform the assignments, output the result into the expected files and exit.
        \end{description}
    \item \textbf{-q, -{}-query} 
        \begin{description}
            \item Search for available classrooms, output the results into the expected files and exit.
        \end{description}
    \item \textbf{-t, -{}-transform} 
        \begin{description}
            \item Transform the School files into compatible files, output the generated files and exit.
        \end{description}
\end{description}


\subsection*{REGULAR EXPRESSIONS}

Three types of regular expressions designed for this project can be used in the \textit{PREFERENCES} and \textit{RESTRICTIONS} files. The three types are taken into account when using them in the name section of a group's code (\textit{subject.type.\textbf{name}}).

The first type involves using the asterisk (*) to indicate that the preference or restriction applies to both English and Spanish groups of a particular type of class (e.g. CVVS.L.*).

The second type uses the addition (+) to indicate that the preference or restriction should apply only to the Spanish groups (e.g. CVVS.L.+).

The last type employs the question mark (?) in the same way as the second type except that the groups concerned are English groups (e.g. CVVS.L.?).


\subsection*{EXIT STATUS}

The program can terminate in two states: \textit{OK} and \textit{ERROR}. The OK status indicates that the execution was completed without problems, while the ERROR status implies that the operation could not be carried out as expected due to an error. The detailed error is recorded in the \textit{LOG} file generated by the utility.


\subsection*{NOTES}

Make sure that all CSV files have a header with the column names, because the program expects such a header to exist. Otherwise it will skip the first row of the CSV. Also, the CSV files cannot have empty rows or the utility will complain. Finally, the utility expects the separator of \textit{all} CSVs to \textit{always} be the semicolon.


\subsection*{COPYRIGHT}

classmanager - manages the classrooms and groups of the School of Computing Engineering of Oviedo.
Copyright (C) 2022  Hugo Fonseca Díaz

This program is free software; you can redistribute it and/or modify
it under the terms of the GNU General Public License as published by
the Free Software Foundation; either version 2 of the License, or
(at your option) any later version.

This program is distributed in the hope that it will be useful,
but WITHOUT ANY WARRANTY; without even the implied warranty of
MERCHANTABILITY or FITNESS FOR A PARTICULAR PURPOSE.  See the
GNU General Public License for more details.

You should have received a copy of the GNU General Public License along with this program; if not, write to the Free Software Foundation, Inc., 51 Franklin Street, Fifth Floor, Boston, MA 02110-1301 USA.


\subsection*{BUGS}


Occasionally an error message will be displayed \textit{after} the message indicating that the program has terminated with error status. It should appear \textit{before} this message. However, repeating the execution may make it appear where expected. 

If you find a bug, you can create an Issue in \href{https://github.com/fonsecadh/classroom-manager-code}{this repository} or send an email to UO258318 AT uniovi DOT es.


\subsection*{EXAMPLES}

Show help:

\verb|java -jar classmanager.jar -h|

\verb|java -jar classmanager.jar --help|

Show version:

\verb|java -jar classmanager.jar -v|

\verb|java -jar classmanager.jar --version|

Performing the assignments:

\verb|java -jar classmanager.jar -a algFolder/algorithm.properties|

\verb|              ioFolder/io.properties|

Note that the user has decided to split the properties in two files, but they could have used only one if they wanted.

Another way of performing the assignments:

\verb|java -jar classmanager.jar --algorithm all.properties|

Searching for classrooms:

\verb|java -jar classmanager.jar -q classfinder.properties|

\verb|java -jar classmanager.jar --query one.properties|

\verb|              two.properties three.properties|

Transforming the files of the School into the files used by the utility:

\verb|java -jar classmanager.jar -t automation.properties|

\verb|java -jar classmanager.jar --transform automation.properties|



\subsection*{SEE ALSO}

You can view the updated documentation at \href{https://github.com/fonsecadh/classroom-manager-doc}{this repository}. If you wish to view the code, you can do so by visiting \href{https://github.com/fonsecadh/classroom-manager-code}{this other repository}.



\section{User manual}

This manual explains the different files involved in the three possible use cases. See Annex \ref{annex-file-format} for examples of each use case.


\subsection*{PERFORM THE ASSIGNMENTS}


\subsubsection*{CONFIG}

A default configuration is provided for the greedy algorithm and for the genetic algorithm. The only notable thing about this property file is that you can specify the path to the optional files but disable their use with boolean keys.

\subsubsection*{INPUT}

Files submitted for classes and subjects can be reused. It is important to keep the headers of each CSV file and not to have empty rows, as discussed in the execution manual. 

If a group or subject code is referenced in another csv file, a check will be made to see if that code exists. However, the codes will not be checked to see if they are well-formed in the original files, so special care must be taken.

The format of dates and times should also be taken into account, so it is recommended to study the files already submitted for reference. The utility will check if they are incorrectly formatted and will warn the user of any problems.

If you start from an existing mapping file, note that there can be no classless mappings or the utility will interpret this as an error when parsing the file.


\subsubsection*{OUTPUT}

Several types of files are generated. It can be categorised into three types, the csv file of assignments, the txt file with the summary of assignments and the files with the timetable in tabular form.

The summary file indicates the assignments sorted by course and subject. The timetables show in table format the timetable of each class. If there is more than one group in a cell, it means that both groups attend classes at the same time, but it is understood that they never overlap in their academic weeks.


\subsection*{SEARCH FOR FREE CLASSROOMS}


\subsubsection*{INPUT}

As the queries are entered via CSV, as many queries as the user wants can be introduced. Care should be taken not to enter a start date that is later than the end date.

\subsubsection*{OUTPUT}

A txt file is generated. This file contains, for each query, its parameters and results. The results include a class, a week, a day and a series of ranges of free slots for that classroom.



\subsection*{AUTOMATICALLY CREATE INPUT FILES}


\subsubsection*{INPUT}

Input files must be correctly adapted to the system. This involves using the semicolon separator and deleting all empty lines. A header must also be created for both files (see example files provided in the Annex).

\subsubsection*{OUTPUT}

The group, group schedules and group academic weeks are generated. Since the system does not contain the subjects in memory for this use case, the entries are sorted by the code of the groups.




\section{Programmer manual}


\subsection*{LOG}

\begin{lstlisting}[basicstyle=\tiny]
DETALLADO: START Persistence logic
jul 05, 2022 12:23:15 AM main.Program main
GRAVE: Wrong integer in GROUPS csv file (cero or negative: 0), line 42
business.errorhandler.exceptions.InputValidationException: Wrong integer in GROUPS csv file (cero or negative: 0), line 42
	at persistence.problem.csv.utils.ValidationUtils.validatePositiveInt(ValidationUtils.java:56)
	at persistence.problem.csv.GroupsDataAccessCsv.validate(GroupsDataAccessCsv.java:106)
	at persistence.problem.csv.GroupsDataAccessCsv.lineToGroup(GroupsDataAccessCsv.java:51)
	at persistence.problem.csv.GroupsDataAccessCsv.loadGroups(GroupsDataAccessCsv.java:29)
	at main.Program.executeExperiments(Program.java:855)
	at main.Program.main(Program.java:94)

jul 05, 2022 12:24:50 AM main.Program main
DETALLADO: START Persistence logic
jul 05, 2022 12:24:50 AM main.Program main
GRAVE: Non existing code for group in PREFERENCES csv file (CPM.L.+), line 6
business.errorhandler.exceptions.InputValidationException: Non existing code for group in PREFERENCES csv file (CPM.L.+), line 6
	at persistence.problem.csv.PreferencesDataAccessCsv.lineToPreferences(PreferencesDataAccessCsv.java:97)
	at persistence.problem.csv.PreferencesDataAccessCsv.loadPreferences(PreferencesDataAccessCsv.java:34)
	at main.Program.executeExperiments(Program.java:883)
	at main.Program.main(Program.java:94)

jul 05, 2022 12:25:21 AM main.Program main
DETALLADO: START Persistence logic
jul 05, 2022 12:25:21 AM main.Program main
DETALLADO: END Persistence logic
jul 05, 2022 12:25:21 AM main.Program main
DETALLADO: START Business logic
jul 05, 2022 12:25:21 AM business.alg.gen.logic.GeneticAlgorithm geneticAlgorithm
DETALLADO: START Genetic Algorithm
jul 05, 2022 12:25:21 AM business.alg.gen.logic.GeneticAlgorithm geneticAlgorithm
MUY DETALLADO: Parameters:

	-> Max number of generations: 10
	-> Max time (ms): 360000
	-> Mutation probability: 0.5
	-> Crossover probability: 0.9
	-> Population size: 200
	-> Fitness function: business.alg.gen.logic.fitness.DefaultFitnessFunction
	-> Individual length: 311
jul 05, 2022 12:25:29 AM business.alg.gen.logic.GeneticAlgorithm geneticAlgorithm
DETALLADO: END Genetic Algorithm
jul 05, 2022 12:25:29 AM main.Program main
DETALLADO: END Business logic
\end{lstlisting}

The above is an extract from the log file. We can see a trace of the flow of the application in three different executions. The first two fail, each with a different error, in the groups and preferences files. The last one occurs after the errors have been fixed and the parameters of the genetic algorithm used for the execution can be observed. Once the execution of the genetic is finished, an entry indicating this fact and another one reporting the completion of the business logic are created.



\subsection*{UTILITIES}

CSV and problem domain validation utilities can be used to lighten and de-code new implementations.

If a new exception type is to be introduced, it must be defined in the business layer error handler package and specified as an expected exception by instructing the ErrorType.



\subsection*{NOTES}

Some notes follow.

\begin{itemize}
    \item New use cases should be executed in the Program class as a unique function with a series of steps (see the other use cases).
    \item All printing to the terminal must be via the CommandLineInterface class.
    \item Changes to the problem domain classes and to the utility classes should be reviewed carefully, as much of the system depends on such classes. 
    \item If more GA operators are to be developed, a Strategy Design Pattern approach, similar to the FitnessFunction, is recommended.
    \item New fitness values for the GA should follow the current procedure of returning a value between 0 and 100.
    \item If data collections are shared between objects, copies of the collections must be shared. If the objects in these collections are to be modified, the shared collection must contain cloned objects (this can be seen in the Decoder class).
\end{itemize}


