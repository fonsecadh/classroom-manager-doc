\renewcommand{\documentname}{System manuals}

\chapter{System manuals}


\section{Installation manual}



\section{Execution manual}


\subsection*{NAME}

\begin{description}
    \item classmanager - manages the classrooms and groups of the School of Computing Engineering of Oviedo.
\end{description}


\subsection*{SYNOPSIS}

\begin{description}
    \item \textbf{classmanager} \textit{OPTION} [\textit{FILE...}]
\end{description}


\subsection*{DESCRIPTION}

The \textit{classmanager} utility assigns classes to groups in the School for a given semester. In addition, it allows queries to be made on the classrooms to obtain those available in a range of dates and schedules. Finally, the utility has a built-in tool that allows for automatic conversion of the files currently used by the school into files compatible with the system. 

A \textit{FILE} represents a configuration file. Each option requires one or more configuration files. If a file is not provided, or if the file cannot be parsed due to lack of permissions, the utility will notify the user of the error. The same goes for the required data not submitted in the configuration file. 


\subsection*{OPTIONS}

\subsubsection*{Generic Program Information}

\begin{description}
    \item \textbf{-h, -{}-help} 
        \begin{description}
            \item Output a usage message and exit.
        \end{description}
    \item \textbf{-v, -{}-version} 
        \begin{description}
            \item Output the version of \textbf{classmanager} and exit.
        \end{description}
\end{description}

\subsubsection*{Functionalities}

\begin{description}
    \item \textbf{-a, -{}-algorithm} 
        \begin{description}
            \item Perform the assignments, output the result into the expected files and exit.
        \end{description}
    \item \textbf{-q, -{}-query} 
        \begin{description}
            \item Search for available classrooms, output the results into the expected files and exit.
        \end{description}
    \item \textbf{-t, -{}-transform} 
        \begin{description}
            \item Transform the School files into compatible files, output the generated files and exit.
        \end{description}
\end{description}


\subsection*{REGULAR EXPRESSIONS}


\subsection*{EXIT STATUS}


\subsection*{NOTES}


\subsection*{COPYRIGHT}


\subsection*{BUGS}


\subsection*{EXAMPLES}


\subsection*{SEE ALSO}



\section{User manual}



\section{Programmer manual}

