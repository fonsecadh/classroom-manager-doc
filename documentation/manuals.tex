\renewcommand{\documentname}{System manuals}

\chapter{System manuals}


\section{Installation manual}



\section{Execution manual}


\subsection*{NAME}

\begin{description}
    \item classmanager - manages the classrooms and groups of the School of Computing Engineering of Oviedo.
\end{description}


\subsection*{SYNOPSIS}

\begin{description}
    \item \textbf{classmanager} \textit{OPTION} [\textit{FILE...}]
\end{description}


\subsection*{DESCRIPTION}

The \textit{classmanager} utility assigns classes to groups in the School for a given semester. In addition, it allows queries to be made on the classrooms to obtain those available in a range of dates and schedules. Finally, the utility has a built-in tool that allows for automatic conversion of the files currently used by the school into files compatible with the system. 

A \textit{FILE} represents a configuration file. Each option requires one or more configuration files. If a file is not provided, or if the file cannot be parsed due to lack of permissions, the utility will notify the user of the error. The same goes for the required data not submitted in the configuration file. 


\subsection*{OPTIONS}

\subsubsection*{Generic Program Information}

\begin{description}
    \item \textbf{-h, -{}-help} 
        \begin{description}
            \item Output a usage message and exit.
        \end{description}
    \item \textbf{-v, -{}-version} 
        \begin{description}
            \item Output the version of \textbf{classmanager} and exit.
        \end{description}
\end{description}

\subsubsection*{Functionalities}

\begin{description}
    \item \textbf{-a, -{}-algorithm} 
        \begin{description}
            \item Perform the assignments, output the result into the expected files and exit.
        \end{description}
    \item \textbf{-q, -{}-query} 
        \begin{description}
            \item Search for available classrooms, output the results into the expected files and exit.
        \end{description}
    \item \textbf{-t, -{}-transform} 
        \begin{description}
            \item Transform the School files into compatible files, output the generated files and exit.
        \end{description}
\end{description}


\subsection*{REGULAR EXPRESSIONS}

Three types of regular expressions designed for this project can be used in the \textit{PREFERENCES} and \textit{RESTRICTIONS} files. The three types are taken into account when using them in the name section of a group's code (\textit{subject.type.\textbf{name}}).

The first type involves using the asterisk (*) to indicate that the preference or restriction applies to both English and Spanish groups of a particular type of class (e.g. CVVS.L.*).

The second type uses the addition (+) to indicate that the preference or restriction should apply only to the Spanish groups (e.g. CVVS.L.+).

The last type employs the question mark (?) in the same way as the second type except that the groups concerned are English groups (e.g. CVVS.L.?).


\subsection*{EXIT STATUS}

The program can terminate in two states: \textit{OK} and \textit{ERROR}. The OK status indicates that the execution was completed without problems, while the ERROR status implies that the operation could not be carried out as expected due to an error. The detailed error is recorded in the \textit{LOG} file generated by the utility.


\subsection*{NOTES}

Make sure that all CSV files have a header with the column names, because the program expects such a header to exist. Otherwise it will skip the first row of the CSV. Also, the CSV files cannot have empty rows or the utility will complain.


\subsection*{COPYRIGHT}

classmanager - manages the classrooms and groups of the School of Computing Engineering of Oviedo.
Copyright (C) 2022  Hugo Fonseca Díaz

This program is free software; you can redistribute it and/or modify
it under the terms of the GNU General Public License as published by
the Free Software Foundation; either version 2 of the License, or
(at your option) any later version.

This program is distributed in the hope that it will be useful,
but WITHOUT ANY WARRANTY; without even the implied warranty of
MERCHANTABILITY or FITNESS FOR A PARTICULAR PURPOSE.  See the
GNU General Public License for more details.

You should have received a copy of the GNU General Public License along with this program; if not, write to the Free Software Foundation, Inc., 51 Franklin Street, Fifth Floor, Boston, MA 02110-1301 USA.


\subsection*{BUGS}


Occasionally an error message will be displayed \textit{after} the message indicating that the program has terminated with error status. It should appear \textit{before} this message. However, repeating the execution may make it appear where expected. 

If you find a bug, you can create an Issue in \href{https://github.com/fonsecadh/classroom-manager-code}{this repository} or send an email to UO258318 AT uniovi DOT es.


\subsection*{EXAMPLES}

TODO


\subsection*{SEE ALSO}

You can view the updated documentation at \href{https://github.com/fonsecadh/classroom-manager-doc}{this repository}. If you wish to view the code, you can do so by visiting \href{https://github.com/fonsecadh/classroom-manager-code}{this other repository}.



\section{User manual}



\section{Programmer manual}

