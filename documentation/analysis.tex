\renewcommand{\documentname}{Analysis}

\chapter{Analysis}


\section{System definition}


\section{System requirements}

\textbf{TODO: THESE ARE THE INITIAL REQUIREMENTS, THEY WILL CHANGE}

The requirements listed here are a basic overview of the fundamental functionality covered by the project. For the complete list of in-depth requirements the reader might refer to NOPE.

\subsection{Interface}

\begin{itemize}
    \item The program must implement a CLI.
        \begin{itemize}
            \item The CLI must show basic or complete information to the user depending on the given option flag.
            \item The CLI must show the encountered errors to the user before terminating the execution.
            \item The CLI must have help, license and version options.
        \end{itemize}
\end{itemize}

\subsection{Input}

\begin{itemize}
    \item The progam receives as input the classrooms, groups, group schedule and the academic weeks of each group.
    \item The program might optionally receive as input a subset of assignments already performed.
    \item The program might optionally receive as input a previous complete list of assignments but without some of the classrooms/laboratories used in it.
    \item The program might optionally receive as input a previous complete list of assignments but with more or less groups.
    \item The program might optionally recieve as input a list of classroom preferences for the groups of a particular subject, given their type (theory or laboratory) and language (english or spanish).
\end{itemize}

\subsection{Configuration}

\begin{itemize}
    \item Program configuration must allow the user to control the parameters of the genetic algorithm.
    \item Program configuration must allow the user to change the version of the program.
    \item Program configuration must allow the user to specify the folder paths for the log and output files.
    \item Program configuration can change in the middle of the course.
\end{itemize}

\subsection{Algorithm}

\begin{itemize}
    \item The program must use a genetic algorithm guided by a greedy algorithm.
    \item Language group requirements:
        \begin{itemize}
            \item English groups should go to different classrooms/laboratories from the spanish groups.
        \end{itemize}
    \item Classroom requirements:
        \begin{itemize}
            \item Some initial classroom assignments can be specified before the execution of the program and they must remain the same.
            \item The program must be able to find a gap in the current list of assignments to include a (mono/multi)-(classroom/laboratory).
            \item The number of groups of the same number and course assigned to the same theory classroom must be maximised.
            \item The number of groups of the same subject assigned to the same laboratory must be maximised.
            \item In each time slot there must be a minimum number of free laboratories.
            \item Some big laboratories must be empty for emergency reasons.
            \item The program must penalise assignments where the number of students is far below the number of computers.
            \item The laboratories must have some free space defined by the user.
            \item In small laboratories (of 16 computers) there must be at least two free computers.
            \item The program must be able to handle a split in two of a laboratory group with only one professor (for emergencies).
        \end{itemize}
\end{itemize}


\section{Subsystem mapping}

\section{Preliminary class diagram}

\section{Analysis of use cases}

\section{Analysis of user interfaces}

\section{Test plan specification}

