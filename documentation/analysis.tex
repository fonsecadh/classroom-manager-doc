\renewcommand{\documentname}{Analysis}

\chapter{Analysis}


\section{System definition}

The prototype defined in this document consists of a command line application which receives a list of text files as input and, by processing them, can perform three operations. The first and main one is the calculation of all the assignments required for the semester, starting from scratch or by using a previous list of assignments generated by the system itself. The second operation consists of finding holes in the schedule, that is, to find free classrooms which comply with a set of constraints. This is useful for assigning a class to a specific event even if the exact time or day of such an event is not known and can only be guessed. The last operation is the automation of the creation of the input files, which is done by converting some other files previously used by the School into new files which the system is able to process.



\section{System requirements}

The functional and non-functional requirements of the system. The non-functional requirements include the technological, system manuals and response time requirements.

\subsection{Functional requirements}

\begin{description}

    \item \textbf{RCLI} The system must implement a command line interface (CLI).
        \begin{description}
            \item \textbf{RCLI1} The CLI must show information about the current process being executed.
            \item \textbf{RCLI2} The CLI must show information about the GA.
                \begin{description}
                    \item \textbf{RCLI2.1} About the parameters of the GA.
                    \item \textbf{RCLI2.2} About the generations of the GA.
                    \item \textbf{RCLI2.3} About the execution time of the GA.
                \end{description}
            \item \textbf{RCLI3} The CLI must show information about the result of the execution.
                \begin{description}
                    \item \textbf{RCLI3.1} If the system terminated successfully.
                    \item \textbf{RCLI3.2} If the system terminated with errors, they will be also notified to the user.
                \end{description}
        \end{description}

    \item \textbf{RInput1} The system must receive the required data as input.
        \begin{description}
            \item \textbf{RInput1.1} The system must receive as input the classrooms of the School.
            \item \textbf{RInput1.2} The system must receive as input the groups of the semester.
            \item \textbf{RInput1.3} The system must receive as input the schedule of the groups.
            \item \textbf{RInput1.4} The system must receive as input the weeks in which the groups have classes.
            \item \textbf{RInput1.5} The system must receive as input the subjects of the semester.
            \item \textbf{RInput1.6} The system must receive as input the queries with the constraints for finding free classrooms in the schedule.
            \item \textbf{RInput1.7} The system must receive as input the previously used files for the automation of the system input files.
        \end{description}

    \item \textbf{RInput2} The system might receive optional data as input.
        \begin{description}
            \item \textbf{RInput2.1} The system might receive as input a total or partial list of assignments.
            \item \textbf{RInput2.2} The system might receive as input a list of classroom preferences.
            \item \textbf{RInput2.3} The system might receive as input a list of classroom restrictions.
        \end{description}

    \item \textbf{RInput3} The system must receive the required configuration files as input.
        \begin{description}
            \item \textbf{RInput3.1} The system must receive as input a configuration file with the parameters of the GA.
            \item \textbf{RInput3.2} The system must receive as input a configuration file with the path to the input and output files.
        \end{description}

    \item \textbf{RConf} The system must be configured by plain text files.
        \begin{description}
            \item \textbf{RConf1} System configuration must allow the user to control the parameters of the GA.
            \item \textbf{RConf2} System configuration must allow the user to specify the paths of the input files.
            \item \textbf{RConf3} System configuration must allow the user to specify the folder paths for the output files.
        \end{description}

    \item \textbf{RAssign} The system must perform the assignments by using AI algorithms.
        \begin{description}
            \item \textbf{RAssign1} The assignments should prioritise that Spanish and English groups go to different classes. 
            \item \textbf{RAssign2} The assignments may start from an initial set of assignments which must remain the same.
            \item \textbf{RAssign3} The assignments must maximize the number of groups of the same name and course assigned to the same theory classroom.
            \item \textbf{RAssign4} The assignments must maximize the number of groups of the same subject assigned to the same laboratory.
            \item \textbf{RAssign5} The assignments must leave a number of free laboratories in each time slot.
            \item \textbf{RAssign6} The assignments should prioritise that a laboratory does not end up with a large number of unused computers.
        \end{description}

    \item \textbf{RClassFinder} The system must be able to find free classrooms for an event given some constraints.
        \begin{description}
            \item \textbf{RClassFinder1} The constraints must include the date range for the search.
            \item \textbf{RClassFinder2} The constraints must include the range of hours for the search. 
            \item \textbf{RClassFinder3} The constraints must include the duration of the event (in hours) for the search.
            \item \textbf{RClassFinder4} The constraints must include the number of attendants to the event. 
            \item \textbf{RClassFinder5} The constraints must include the type of classroom to hold the event in. 
            \item \textbf{RClassFinder6} The constraints must include the maximum number of results to obtain. 
        \end{description}

    \item \textbf{RAutomation} The system must be able to automate the creation of the input files.
        \begin{description}
            \item \textbf{RAutomation1} The system must receive the planning for the semester.
            \item \textbf{RAutomation2} The system must receive the number of enrolled students for each group.
        \end{description}

    \item \textbf{RLog} The system must keep a log of its operations.
        \begin{description}
            \item \textbf{RLog1} The log must indicate the date and time of every record.
            \item \textbf{RLog2} The log must indicate the log level of every record.
            \item \textbf{RLog3} The log must record the complete information of encountered errors.
            \item \textbf{RLog4} The log must record basic information of the flow of the application.
        \end{description}

\end{description}


\subsection{Non-Functional requirements}

\begin{description}

    \item \textbf{RTech} The system requires a specific setup to be executed.
        \begin{description}
            \item \textbf{RTech1} The system requires Java 8 to be installed in the computer which executes it.
            \item \textbf{RTech2} The system requires that the folders where the input files are located have sufficient permissions for the system to be able to read these files.
        \end{description}

    \item \textbf{RMan} The system manuals must provide the readers with appropriate information for carrying out their tasks.  
        \begin{description}
            \item \textbf{RMan1} The installation manual must explain the setup needed before the execution of the program.
                \begin{description}
                    \item \textbf{RMan1.1} It is aimed both at the user and the developers of the application.
                \end{description}
            \item \textbf{RMan2} The execution manual must explain the syntax for executing each functionality of the system.
                \begin{description}
                    \item \textbf{RMan2.1} It is aimed both at the user and the developers of the application.
                \end{description}
            \item \textbf{RMan3} The user manual must explain all the functionality of the system.
                \begin{description}
                    \item \textbf{RMan3.1} It is aimed at the user of the application.
                    \item \textbf{RMan3.2} It must provide examples of usage with step by step instructions. 
                \end{description}
            \item \textbf{RMan4} The programmer manual must briefly explain the structure of the code and its components.
                \begin{description}
                    \item \textbf{RMan4.1} It is aimed at the developers and maintainers of the application.
                    \item \textbf{RMan4.2} It must provide examples of possible changes with some directions to implement them. 
                    \item \textbf{RMan4.3} It must provide an explanation on how to interpret the log of the application. 
                \end{description}
        \end{description}

    \item \textbf{RResponse} The system must perform the assignments of a semester in less than a day.
        \begin{description}
            \item \textbf{RResponse1} The assignments must have the expected quality. An assignment is said to be of quality if it meets all hard and as many soft constraints as possible.
        \end{description}

\end{description}



\section{Subsystem mapping}

\section{Preliminary class diagram}

\section{Analysis of use cases}

\section{Analysis of user interfaces}

\section{Test plan specification}

