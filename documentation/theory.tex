\renewcommand{\documentname}{Theoretical Aspects}

\chapter{Theoretical aspects}\label{theory}


\section{Assignment problem}

Imagine that a digital magazine tasked you with the following problem. They need to assign book reviews to freelance writers, and they want the assignments to result in the maximum profit. Therefore, we have:

\begin{description}
    \item A set of $n$ freelancers $f$
    \item A set of $m$ book reviews $r$
    \item A set of $n \times m$ assignments $a_{fr}$ such that $a_{fr} = 0$ when freelancer $f$ is not assigned to book review $r$ and $a_{fr} = 0$ when freelancer $f$ is assigned to book review $r$.
    \item A set of $n \times m$ profits $p_{fr}$ which indicate the profit obtained when assigning freelancer $f$ to book review $r$ and that $p_{fr} \textgreater 0$.
    \item A valid solution is defined as a set of assignments where all the book reviews have a freelancer assigned to them and no book review has more than one associated freelancer.
\end{description}

The profit for all the assignments will then be:

\begin{equation}
    \scalebox{2}{$\sum_{f=1}^n \sum_{r=1}^m a_{fr} p_{fr}$}
\end{equation}

The optimal solution consists on having a set of assignments such that the sum of all the profits for the current assignments is maximised.

For example, let us say that we have two freelancers $f1$ and $f2$ and three book reviews $r1$, $r2$ and $r3$. We can represent this data with two sets $F$ and $R$, respectively.

\begin{align}
    F &= \{ f1, f2 \} \\
    R &= \{ r1, r2, r3 \}
\end{align}

Then, our assignments and profits will be represented by the sets $A$ and $P$.

\begin{align}
    A &= \{ a11, a12, a13, a21, a22, a23 \} \\
    P &= \{ p11, p12, p13, p21, p22, p23 \}
\end{align}

Now, we are going to study valid and non-valid solutions. As we explained before, a solution is valid if every book review has a freelancer assigned to it, and no more than one.

We will analyse four sets of values for the A set:

\begin{align}
    A1 &= \{ 0, 0, 0, 1, 1, 1 \} \\
    A2 &= \{ 0, 1, 0, 1, 0, 1 \} \\
    A3 &= \{ 0, 0, 0, 1, 0, 1 \} \\
    A4 &= \{ 0, 1, 0, 1, 1, 1 \}
\end{align}

From these sets, we can deduce that $A1$ and $A2$ are valid solutions, because they have one freelancer for each book review. We are not concerned with a freelancer having no book reviews assigned. However, a book review without an associated freelancer represents a non-valid solution. That is precisely the case for $A3$, the book review $r2$ has not an assigned freelancer. In the case of $A4$, the fact that $r2$ has two freelancers assigned to it makes it a non-valid solution.

Now, we will give values to the $P$ set in order to discuss possible optimal solutions. We will compare them with the assignment sets $A1$ and $A2$

\begin{align}
    P1 &= \{ 3, 1, 6, 10, 2, 15 \} \\
    P2 &= \{ 3, 7, 6, 10, 2, 15 \}
\end{align}

$P1$ y $A1$:

\begin{equation}
    \scalebox{1.2}{$Profit = \sum_{f=1}^n \sum_{r=1}^m a_{fr} p_{fr} = 1 \times 10 + 1 \times 2 + 1 \times 15 = 27$}
\end{equation}

$P1$ y $A2$:

\begin{equation}
    \scalebox{1.2}{$Profit = \sum_{f=1}^n \sum_{r=1}^m a_{fr} p_{fr} = 1 \times 1 + 1 \times 10 + 1 \times 15 = 26$}
\end{equation}

We can observe that for $P1$, the set of assignments $A1$ is better than $A2$, because it results in a better profit. Another important thing about $A1$ is that it is the optimal solution to the problem, because it assigns the book reviews to the freelancers with the better profit value. Now $P2$ will be evaluated.

$P2$ y $A1$:

\begin{equation}
    \scalebox{1.2}{$Profit = \sum_{f=1}^n \sum_{r=1}^m a_{fr} p_{fr} = 1 \times 10 + 1 \times 2 + 1 \times 15 = 27$}
\end{equation}

$P2$ y $A2$:

\begin{equation}
    \scalebox{1.2}{$Profit = \sum_{f=1}^n \sum_{r=1}^m a_{fr} p_{fr} = 1 \times 7 + 1 \times 10 + 1 \times 15 = 32$}
\end{equation}

In the case of the profits values of $P2$, the situation is reversed. $A2$ is now the optimal solution and therefore better than $A1$.

The important thing to notice here is that the values for the $P$ set right now may appear as having no meaning whatsoever. But if we think of $P$ as a profict funtion, maybe we can interpret $P1$ as values of profit in a context where freelancer $f1$ has just tweeted an opinion on the book related with the review $r2$ and caused a massive controversy. We can then say that $P1$ is a function which gives more importance to public relations and so the profit $p_{12}$ is very low, whereas $P2$ gives more importance to clickbait and so the profit $p_{12}$ is higher! Of course in a real problem you know what the function is calculating, but for this example it is left to the imagination of the reader.

With this example of an assignment problem we have seen its main characteristics and worked around with it to some extent. We will return to it time and time again as we follow through this chapter. 


\section{Heuristics}


\section{Greedy algorithm}


\section{Metaheuristics}


\section{Genetic algorithm}

